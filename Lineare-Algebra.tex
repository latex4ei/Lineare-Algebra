% Mathe Formelsammlung für Lineare Algebra WiSe 23/24
% 2 Seiten

% Dokumenteinstellungen
% ======================================================================	

% Dokumentklasse (Schriftgröße 6, DIN A4, Artikel)
\documentclass[german, 6pt]{latex4ei/latex4ei_sheet}
\title{Lineare Algebra}
\author{Lukas Kompatscher, Paul Thiel, Justus Rossmeier, Darius Peters}
\myemail{info@latex4ei.de}
\mywebsite{www.latex4ei.de}

% Pakete laden
\usepackage{setspace}
\usepackage{tikz}
\usetikzlibrary{matrix, positioning}
  
% Seitenlayout und Ränder:
\titlespacing*{\section}{0pt}{2pt}{2pt}
\titlespacing*{\subsection}{.5pt}{1pt}{2pt}
\setstretch{1.1}
\titleformat*{\section}{\large\bfseries}
\titleformat*{\subsection}{\normalsize\bfseries}
\colorlet{col_section}{black}

% Custom Commands
\newcommand{\me}[1]{\ensuremath{\left\{#1\right\}}}
\newcommand{\dme}[2]{\ensuremath{\left\{#1\,\vert\,#2 \right\}}}
\newcommand{\un}[1]{\; \unit{#1} }
\newcommand{\unf}[2]{\;\left[ \unitfrac{#1}{#2} \right]}
\newcommand{\enbrace}[1]{\ensuremath{\left(#1\right)}}
\newcommand{\nira}[1]{\ensuremath{\overset{n \rightarrow \infty}{\longrightarrow}}}
\newcommand{\os}[2]{\ensuremath{\overset{#1}{#2}}}
\makeatletter
\newcommand{\gk}[1]{\ensuremath{\left\lfloor#1\right\rfloor}}
  
%Custom functions
\DeclareMathOperator{\arccot}{arccot}
\DeclareMathOperator{\im}{im}
\DeclareMathOperator{\rang}{rang}
\DeclareMathOperator{\col}{col}
\DeclareMathOperator{\row}{row}
\DeclareMathOperator{\Kern}{kern}
\DeclareMathOperator{\proj}{proj}

%Custom font
\usepackage{lmodern}
\renewcommand{\ttdefault}{lmtt}

% Dokumentbeginn
% ======================================================================
\begin{document}

%\section{}
% ----------------------------------------------------------------------
\maketitle
\section{Allgemeines} % (fold)
\label{sub:allgemeines}

Dreiecksungleichung \qquad \qquad \qquad
\begin{math}\begin{array}{l}
	\abs{x + y} \le \abs{x} + \abs{y}
\end{array}\end{math} \\
Cauchy-Schwarz-Ungleichung: \qquad 
\begin{math}\begin{array}{l}
\left| \sprod{x}{y} \right| \le \| x\| \cdot \| y\|
\end{array}\end{math}\\
$\mathbb{K}$ steht für $\mathbb{R}$ und $\mathbb{C}$\\
$\mathbb{I}_n$ ist die $n\times n$-Einheitsmatrix. \qquad$e_i$ ist der i-te Einheitsvektor.
\section{Matrizen}
% ----------------------------------------------------------------------
Die Matrix $A=(a_{ij}) \in \mathbb K^{m\times n}$ hat $m$ Zeilen mit Index $i$ und $n$ Spalten mit Index $j$.

\subsection{Allgemeine Rechenregeln}
\textbf{Merke:} Zeile vor Spalte! (Multiplikation, Indexreihenfolge, etc.)
\begin{tabular}{ll}	
	1)  $A+0=A$ & 2)  $1 \cdot A=A$ \\
	3)  $A+B=B+A$ & 4) $A \cdot B \ne B \cdot A$ (im Allg.) \\
	5)  $(A+B)+C=A+(B+C)$ & 6) $\lambda (A+B) = \lambda A + \lambda B$\\ 
\end{tabular}\\
Multiplikation von $A\in \mathbb K^{m\times r}$ und $B\in \mathbb K^{r\times n}$: $AB\in\mathbb K^{m\times n}$

\subsection{Elementare Zeilenumformungen (EZF) (gilt äquiv. für Spalten)}
$A \in \mathbb K^{m\times n}$ hat $m$ Zeilen $z_i\in \mathbb K^n$
\begin{itemize}\itemsep0pt
\item Vertauschen von Zeilen
\item Multiplikation einer Zeile mit $\lambda\ne 0$ 
\item Addition des $\lambda$-fachen der Zeile $z_i$ zur Zeile $z_j$
\end{itemize}

\subsection{Transponieren}
$A=(a_{ij})\ \in \mathbb K^{m\times n}$ gilt: $A^\top=(a_{ji})\ \in \mathbb K^{n\times m}$\\
\textbf{Regeln:}\\
$(A+B)^\top=A^\top+B^\top$\qquad $(A\cdot B)^\top=B^\top\cdot A^\top$\qquad \\
$(\lambda A)^\top=\lambda A^\top$ \qquad \qquad \qquad $(A^\top)^\top=A$\\
\\
$A\in \mathbb K^{n\times n}$ ist symmetrisch, falls $A=A^\top$ ($\Rightarrow$ orth. diagbar)\\
$A\in \mathbb K^{n\times n}$ ist schiefsymmetrisch, falls $A=-A^\top$\\
$A\in \mathbb K^{n\times n}$ ist orthogonal (Spalten-/Zeilenvektoren=ONB), falls:\\
$AA^\top=\mathbb{I}_n \quad \Leftrightarrow \quad A^\top=A^{-1} \quad \Leftrightarrow \quad \det A=\pm 1$\\
$A\in \mathbb C^{n\times n}$ ist hermitesch, falls $A=\overline{A}^\top$  \quad (kmplx. konj. u. transp.)


\subsection{Inverse Matrix}
Für die inverse Matrix $A^{-1}$ von $A\in \mathbb K^{n\times n}$ gilt: $A^{-1}A=\mathbb{I}_n$\\
$(A^{-1})^{-1}=A$ \qquad $(AB)^{-1}=B^{-1}A^{-1}$ \\
$(A^\top)^{-1}=(A^{-1})^\top$\\
\\
$A\ \in \mathbb K^{n\times n}$ ist invertierbar, falls: $\det (A) \ne 0 \quad \lor \quad \rang(A)=n$\\
\\
Berechnen von $A^{-1}$ nach Gauß:\\
$AA^{-1}=\mathbb{I}_n\quad\Rightarrow\quad (A|\mathbb{I}_n)\overset{EZF}{\longrightarrow}(\mathbb{I}_n|A^{-1})$\\
2x2-Matrix:$\enbrace{\begin{matrix}
a & b \\
c & d
\end{matrix}}^{-1} = \frac{1}{ad-bc}\begin{pmatrix}
d & -b \\
-c & a
\end{pmatrix}$

\subsection{Rang einer Matrix}
$A\in \mathbb K^{m\times n}$ 
{(N0-Zeilen = Nicht-Null-Zeilen)}\\ \\
\begin{minipage}{\columnwidth}
\textbf{Bringe A auf ZSF} \\
Rang (Zeilrang) $\rang(A)$: Anzahl N0-Zeilen \\     
Zeilenraum $\row(A)$: Erzeugnis der Zeilen, $\text{Basis}(\row(A)) = \{\text{ N0-Zeilen }\}$ \\
Kern: $\Kern(A) = \dme{x \in \mathbb K^n}{Ax= 0}$ \\
Dimensionsformel: $\rang(A) + \mathrm{dim}(\Kern(A)) = n$ \\
\end{minipage}
\begin{minipage}{\columnwidth}
\textbf{Bringe A auf Spaltenstufenform (transponieren, ZSF)} \\
Spaltenrang: Anzahl der N0-Spalten\\
Spaltenraum $\col(A)$: Erzeugnis der Spalten, $\text{Basis}(\col(A)) = \{\text{ N0-Spalten }\}$ \\
Bild = Spaltenraum: Erzeugnis der Spalten 
\end{minipage}
\begin{minipage}{\columnwidth}
\subsection{Matrixpotenzen}
Gegeben: $A \in\mathbb{R}^{mxn}, x \in\mathbb{R}^n$.\\
Gesucht: Lösung von $A^n$.
\begin{itemize}
	\item Bestimme Eigenwerte $\lambda$ und Eigenvektoren $v$ von $A$.
	\item Bestimme $\alpha_1, ..., \alpha_k$ mit $x = \alpha_1v_1 + ... + \alpha_kv_k$.
	\item $A^nx = \alpha_1\lambda_1^nv_1+ ... + \alpha_k\lambda_k^nv_k$.
\end{itemize}
Gegeben: $A \in\mathbb{R}^{nxn}, A$ diagonalisierbar.\\
Gesucht: $A^n$.
\begin{itemize}
	\item $A^n = SD^nS^{-1}$ mit $D^n = \begin{pmatrix}
	\lambda_1^n & \ldots & 0 \\
	\vdots & \ddots & \vdots\\
	0 & \ldots & \lambda_k^n \\
	\end{pmatrix}$.
\end{itemize}
\end{minipage}
\subsection{Lineares Gleichungssystem LGS}
Das LGS $Ax=b$ kurz $(A|b)$ mit $A\in \mathbb K^{m\times n}$, $x\in \mathbb K^n$, $b\in \mathbb K^m$ hat $m$ Gleichungen und $n$ Unbekannte.\\
\\
\textbf{Lösbarkeitskriterium:}\\
Ein LGS $(A|b)$ ist genau dann lösbar, wenn: $\rang(A)=\rang(A|b)$\\
Die Lösung des LGS $(A|b)$ hat $\dim(\Kern(A)) = n-\rang(A)$ frei wählbare Parameter.\\
\\
Das LGS hat eine Lsg. wenn $\det A \not= 0$ \quad $\rightarrow \exists A^{-1}$ \\
Das homogene LGS: $(A|0)$ hat stets die triviale Lösung $0$\\
Summen und Vielfache der Lösungen von $(A|0)$ sind wieder Lösungen.

\subsection{Determinante}
$A\in \mathbb K^{n\times n}$: $\det(A)=|A|$
\begin{itemize}\itemsep0pt
\item $|A|=\sum\limits_{i\text{ oder }j=1}^n (-1)^{i+j} \cdot a_{ij} \cdot |A_{ij}|$ \\
Entwicklung nach $j$-ter Spalte oder $i$-ter Zeile
%\item $|A|=\sum\limits_{j=1}^n (-1)^{i+j} \cdot a_{ij} \cdot |A_{ij}|$ \qquad Entwicklung n. $i$-ter Zeile
\item Determinante einer $2\times 2$-Matrix: \\
$A=\begin{pmatrix}a&b\\c&d\end{pmatrix} \Rightarrow \det(A)=|A|=ad-bc$
\item Determinante einer $3\times 3$-Matrix:
\[\det(A)=\det
\begin{tikzpicture}[baseline=(current bounding box.center), scale=0.4]
    \matrix (m) [matrix of math nodes,left delimiter=(,right delimiter=)] {
        a_{11} & a_{12} & a_{13} \\
        a_{21} & a_{22} & a_{23} \\
        a_{31} & a_{32} & a_{33} \\
    };
    \matrix (extended) [matrix of math nodes, right=0.05cm of m.east, anchor=west] {
        a_{11} & a_{12} \\
        a_{21} & a_{22} \\
        a_{31} & a_{32} \\
    };
    
    \draw[thin, red] (m-1-1.north west) -- (m-3-3.south east);
    \draw[thin, red] (m-1-2.north west) -- (extended-3-1.south east);
    \draw[thin, red] (m-1-3.north west) -- (extended-3-2.south east);
    \draw[thin, blue] (m-1-3.north east) -- (m-3-1.south west);
    \draw[thin, blue] (extended-1-1.north east) -- (m-3-2.south west);
    \draw[thin, blue] (extended-1-2.north east) -- (m-3-3.south west);
\end{tikzpicture}
\]
\resizebox{\columnwidth}{!}{$
    =\textcolor{red!70!white}{a_{11}a_{22}a_{33}+a_{12}a_{23}a_{31}+a_{13}a_{21}a_{32}}%
    \textcolor{blue!70!white}{-a_{12}a_{21}a_{33}-a_{11}a_{23}a_{32}-a_{13}a_{22}a_{31}}
    $}
\item $\det\begin{pmatrix}A&0\\C&D\end{pmatrix}=\det\begin{pmatrix}A&B\\0&D\end{pmatrix}=\det(A)\cdot\det(D)$
\item $\begin{vmatrix}\lambda_1&&* \\ &\ddots& \\ 0&&\lambda_n \end{vmatrix} = \lambda_1\cdot \ldots\cdot \lambda_n = \begin{vmatrix} \lambda_1&&0  \\  &\ddots& \\  *&&\lambda_n \end{vmatrix}$
\item $A=B \cdot C \quad \Rightarrow \quad |A|=|B| \cdot |C|$
\item $\det(A)=\det(A^\top)$
\item Hat $A$ zwei gleiche Zeilen/Spalten $\Rightarrow |A|=0$
\item $\det(\lambda A)=\lambda^n \det(A)$
\item Ist $A$ invertierbar, so gilt: $\det(A^{-1})=(\det(A))^{-1}$
\item $\det(AB) = \det(A) \det(B) = \det(B) \det(A) = \det(BA)$
\end{itemize}
\textbf{Umformung Determinante}
\begin{itemize}\itemsep0pt
\item Vertauschen von Zeilen/Spalten ändert Vorzeichen von $|A|$
\item Zeile/Spalte mit $\lambda$ multiplizieren, $|A|$ um Faktor $\lambda$ größer
\item Addition des $\lambda$-fachen der Zeile X zur Zeile Y ändert $|A|$ nicht 
\end{itemize}
%\textbf{Vereinfachung für Spezialfall $A\in \mathbb K^{2\times 2}$}\\
%$A=\begin{pmatrix}a&b\\c&d\end{pmatrix} \Rightarrow \det(A)=|A|=ad-bc$
\subsection{Ähnlichkeit von Matrizen}
Zwei Matrizen $A, B \in \mathbb{R}^{n\times n}$ sind ähnlich zueinander, wenn es eine invertierbare Matrix $S$ gibt, sodass $A=SBS^{-1}$. Man schreibt $A \sim B$.\\
\textbf{Eigenschaften:}\\
\begin{tabular}{ll}	
	  $A$ ist inviertierbar $\Leftrightarrow$ $B$ ist invertierbar &   $\det(A)=\det(B)$\\
	  $A$ und $B$ haben das gleiche char. Polynom
 &  $\rang(A)=\rang(B)$
\end{tabular}
\subsection{Hauptsatz invertierbarer quadratischer Matrizen}
\begin{tabular}{ll}
	1)  $A$ ist invertierbar & 2) $\dim(\col(A))=\dim(\row(A))=n$\\
	3)  $\Kern(A)={0}$ & 4) Die strenge ZSF von $A$ ist $\mathbb{I}_n$\\
	5) $\det(A)\ne0$ & 6) Zeilen/Spalten von $A$ linear unabh.\\
	7) $Ax=b$ hat eine & 8) $0$ ist kein Singulärwert von $A$\\\ \ \ \ eindeutige Lösung $\forall b\in\mathbb{R}^n$ & 9) Lineare Abbildung ist bijektiv\\
	10)  $\rang(A)=n$ & 11) $0$ ist kein Eigenwert von $A$\\
	12) $\col(A)=\mathbb{R}^n$ & 13) $A$ ist Produkt elementarer Matrizen\\
\end{tabular}\\ \\
\textit{Ist eine dieser Aussagen wahr, so sind alle Aussagen wahr!}
\section{Vektoren}
\begin{minipage}{\columnwidth}
Ein Vektor ist ein $n$-Tupel reeller oder komplexer Zahlen, also ein Element aus dem $\mathbb K^n$.
\subsection{Skalarprodukt}
$\langle v,w \rangle: V\times V \rightarrow \mathbb R$ 
	\begin{enumerate}\itemsep0pt
	\item Linear: $\langle u,v+w \rangle=\langle u,v \rangle + \langle u,w \rangle \land \langle u,\lambda v \rangle = \lambda \langle u, v \rangle$ 
	\item Symmetrisch: $\langle v,w \rangle=\langle w,v \rangle$
	\item Positiv definit: $\langle v,v \rangle\ge0$  \qquad $\langle v,v \rangle=0 \Leftrightarrow v=0$
	\end{enumerate} 
\textbf{Kanonisches Skalarprodukt} \\
$\langle v,w \rangle=v^\top  w=v_1w_1+\dots+v_nw_n$\\ \\
\textbf{Skalarprodukt} bzgl. sym., quadr. und positiv definiter Matrix $A\in \mathbb K^{n\times n}$\\
$\langle v,w \rangle_A=v^\top A w$\\
\textbf{Skalarprodukt Polynome} $\sprod{p(x)}{q(x)}=\int\limits_{0}^{1}p(x)q(x)\,dx$ \\
\textbf{Orthogonalität} $\sprod{a}{b} = 0 \Leftrightarrow a\perp b$ \\
\textbf{Projektion} eines Vektor $v$ längs $a$: $\proj_a(v) = \frac{\sprod{v}{a} }{\sprod{a}{a} }\cdot a$\\
\textbf{Orthogonale Zerlegung} eine Vektors $v$ längs $a$:\\ 
$v = \proj_a(v) + \proj_{a^\perp}(v) \Rightarrow \proj_{a^\perp}(v) = v - \proj_a(v)$\\
\textbf{Winkel} \quad 	$\cos \phi = \frac{\sprod{a}{b}}{\norm{a} \norm{b}} $ \qquad
$\phi = \arccos \enbrace{ \frac{\sprod{a}{b} }{\norm{a} \norm{b} } }$
\end{minipage}
\subsection{Kreuzprodukt (Vektorprodukt)}
$a\times b=\left( \begin{matrix} a_2b_3-a_3b_2\\a_3b_1-a_1b_3\\a_1b_2-a_2b_1\end{matrix}\right)$\qquad $a,b\ \in \mathbb R^3$\\
\\
$a\times b \perp a,b$ \qquad (falls $a\times b=0\ \Leftrightarrow\ a,b$\ linear abhängig)\\
$a \times b = -b \times a$\\
$\norm{a\times b}=\norm{a}\cdot\norm{b}\cdot \sin\left(\measuredangle (a,b)\right)\mathrel{\widehat{=}}$\ Fläche des Parallelogramms\\
Graßmann-Identität: $a\times(b \times c)\equiv b\cdot(a \cdot c)-c\cdot(a \cdot b)$\\
\\
\textbf{Spatprodukt}\\
$[a,b,c]:=\langle  a\times b, c\rangle=\det (a,b,c)\mathrel{\widehat{=}}$\ Volumen des Spates.\\
$[a,b,c]>0\ \Leftrightarrow\ a,b,c$\ bilden Rechtssystem \\ $[a,b,c]=0\ \Leftrightarrow\ \{a,b,c\}$\ linear abhängig

\section{Vektorräume (VR)}
% --------------------------------------------------------------
Eine nichtleere Menge V mit zwei Verknüpfungen $+$ und $\cdot$ heißt $K$-Vektorraum über dem Körper $\mathbb K$. \\
\textbf{Bedingung $(u,v,w\in V \quad \lambda,\mu \in \mathbb{R})$}\\ \\
\begin{tabular}{ll}
\begin{minipage}{.5\columnwidth}
\begin{itemize}\itemsep0pt
\item $v+w\in V$ \qquad $\lambda v \in V$
\item $u+(v+w)=(u+v)+w$
\item $0\in V: v+0=v$
\item $v'\in V: v+v'=0$
\end{itemize}
\end{minipage}
&
\begin{minipage}{.5\columnwidth}
\begin{itemize}\itemsep0pt
\item $v+w=w+v$
\item $\lambda(v+w)=\lambda v + \lambda w$
\item $(\lambda + \mu)v=\lambda v +\mu v $
\item $(\lambda \mu)v = \lambda(\mu v)$
\end{itemize}\itemsep0pt
\end{minipage}
\end{tabular}
\subsection{Untervektorraum (UVR)}
$U\subset V (u,v\in U \quad \lambda\in\mathbb{R})$
\begin{enumerate}\itemsep0pt
\item $U\ne \emptyset \qquad (0\in U)$
\item $u+v\in U$
\item $\lambda u \in U$
\end{enumerate}

\subsection{Basis (Jeder VR und jeder UVR besitzt eine Basis!)} % (fold)
\label{sub:basis}
 Eine Teilmenge $B\subset V$ heißt Basis von $V$, wenn gilt:
\begin{itemize}\itemsep0pt
	\item $\mathrm{span}(B) =V$, $B$ erzeugt $V$
	\item $B$ ist linear unabhängig
\end{itemize}    

\subsection{Dimension}
$n=\dim(V)=\abs{B} = $ Mächtigkeit von $B$\\
Mehr als $n$ Vektoren aus $V$ sind stets linear abhängig. \\
Für jeden UVR $U \subset V$ gilt: $\dim (U) \le \dim (V)$ 

\subsection{Linearkombination}
Jeder Vektor $v\in\mathbb{K}^n$ kann als Linearkombination einer Basis $B=\{b_1, \dots, b_n\} \subset \mathbb{K}^n$ dargestellt werden
\begin{equation*}
v=\lambda_1 b_1 + \dots + \lambda_n b_n \Rightarrow \text{Gauß} \ \Big(b_1 \ b_2 \ b_3 \ |\ v\ \Big)
\end{equation*}
\textbf{Linear Unabhängig:}
Vektoren heißen linear unabhängig, wenn aus: \\
$\lambda_1 v_1 + \dots + \lambda_n v_n = 0$ folgt, dass $\lambda_1 = \dots = \lambda_n = 0$

\subsection{Orthogonalität}
$B\subset V$ heißt
\begin{itemize}\itemsep0pt
\item \textbf{Orthogonalsystem}, wenn $\forall v,w\in B: v\perp w$
\item \textbf{Orthogonalbasis}, wenn $B$ Orthogonalsystem und Basis von $V$
\item \textbf{Orthonormalsystem}, wenn $B$ Orthogonalssystem u. $\forall v\in B: \norm{v}=1$
\item \textbf{Orthonormalbasis(ONB)},  wenn $B$ Orthonormalsystem u. Basis von V ist
\end{itemize}
Eine \textbf{quadratische Matrix $A$ heißt orthogonal}, wenn $A^\top A = \mathbb{I}_n$
\begin{tabular}{p{0.45\linewidth}p{0.45\linewidth}}
\begin{itemize}\itemsep6pt
\item $A^{-1}=A^\top $
\item $\det{A}=\pm1$
\item Spalten bilden ONB
\item Zeilen bilden ONB
\item $\|Av\|=\|v\|$ % Using \norm for the norm
\end{itemize}
&
\begin{itemize}
\item Drehmatrix mit Drehung um den Ursprung:\ \
$\begin{pmatrix}
\cos(\alpha) & -\sin(\alpha)\\
\sin(\alpha) & \cos(\alpha)
\end{pmatrix}$
\item (Dreh-)Spiegelmatrix mit Spiegelung an der Geraden $y=tan(\frac{\alpha}{2})\cdot x$:\ \
$\begin{pmatrix}
\cos(\alpha) & \sin(\alpha)\\
\sin(\alpha) & -\cos(\alpha)
\end{pmatrix}$
\end{itemize}
\end{tabular}
\textbf{Orthonormalisierungsvefahren einer Basis $\{v_1,\ldots,v_n\}$ nach Gram-Schmidt}
\begin{enumerate}\itemsep0pt
\item $b_1=\frac{c_1}{\|c_1\|}$\ \ mit \ \ $c_1=v_1$ \ \ (Vektor mit vielen 0en oder 1en)
\item $b_{2}= \frac{c_2}{\norm{c_2}}$\ \ mit \ \ $c_2=v_2-\frac{\sprod{v_2}{c_1}}{\sprod{c_1}{c_1}}\cdot c_1$
\item $b_{3}= \frac{c_3}{\norm{c_3}}$\ \ mit \ \ $c_3=v_3-\frac{\sprod{v_3}{c_1}}{\sprod{c_1}{c_1}} \cdot c_1-\frac{\sprod{v_3}{c_2}}{\sprod{c_2}{c_2}}\cdot c_2$
\end{enumerate}
\textbf{Erweitern einer ONB von $V$ auf eine ONB des $\mathbb{R}^n$}
\begin{enumerate}\itemsep0pt
	\item Vektor $e_i \text{ mit } i \in \{1...n\}$ so wählen, sodass die Skalarprodukte möglichst einfach zu berechnen sind.
	\item Gram-Schmidt für $e_i$ $\Rightarrow$ Ergebnis zur Basis hinzufügen
	\item So lange Wiederholen bis die Basis $n$ Vektoren besitzt.
	\item Alle neu hinzugefügten Vektoren bilden zusammen eine ONB von $V^\perp$
\end{enumerate}
\textbf{Orthogonale Projektion auf UVR} \\
Gegeben: Vektorraum $V \in \mathbb{R}^n$, $v\in V$, Untervektorraum $U\subset V$
\begin{enumerate}\itemsep0pt
\item Basis von $U$ bestimmen
\item Orthogonalisiere Basis $\{u_1,u_2,u_3,\ldots\}$ von $U$
\item $\proj_U(v) = \frac{\sprod{v}{u_1}}{\sprod{u_1}{u_1}}u_1 + \frac{\sprod{v}{u_2}}{\sprod{u_2}{u_2}}u_2 +\dots$
\item $\proj_{U^\perp}(v)=v-\proj_U(v)$
\item Abstand von $v$ zu $U$ = $\norm{\proj_{U^\perp}(v)}$
\end{enumerate}
Alternative Methode
\begin{enumerate}\itemsep0pt
\item Basis $\{b_1,\ldots,b_r\}$ von $U$ bestimmen
\item Setze $A = \big(b_1\ b_2\ \dots \ b_r\big) \in \mathbb{R}^{n\times r}$
\item Löse das LGS $A^\top Ax=A^\top v$ und erhalte den Lösungsvektor $x=(\lambda_1, \dots,\lambda_r)^\top $
\item $\proj_U(v)=\lambda_1 b_1 +\dots +\lambda_r b_r$
\end{enumerate}

\section{Norm}
\textbf{Euklidische Norm von Vektoren}
$\norm{a}=\sqrt{\sprod{a}{a}} =\sqrt{a_1^2+a_2^2+\ldots +a_n^2}$\\
$\forall v, w \in \mathbb{R}^n$:\\
1. $\norm{\lambda v}=\abs{\lambda}\norm v$
\qquad 2. $\norm{v+w}\le \norm{v} + \norm{w}$

\begin{minipage}{\columnwidth}
\section{Lineare Abbildungen}
Abbildung $f:V\rightarrow W$ ist linear, wenn
\begin{enumerate}\itemsep0pt
\item $f(0)=0$
\item $f(a+b)=f(a)+f(b)$
\item $f(\lambda a)=\lambda f(a)$
\end{enumerate}
$\Rightarrow$ Abbildung als Matrix darstellbar (siehe Abbildungsmatrix)\\ \\
\textbf{Injektiv}, wenn aus $f(x_1)=f(x_2) \Rightarrow x_1=x_2$\\
\textbf{Surjektiv}: $\forall y\in W \ \exists x\in V:f(x)=y$\\ \quad (Alle Werte aus $W$ werden angenommen.)\\
\textbf{Bijektiv}(Eineindeutig): $f$ ist injektiv und surjektiv $\Rightarrow$ $f$ umkehrbar.
\end{minipage}
\subsection{Koordinatenvektor bezüglich einer Basis $B$}
Gegeben: Vektorraum $V \in\mathbb{R}^n$, $v \in V$.\\
Gesucht: $[v]_B$ (Koordinaten von $v$ bezüglich der Basis $B$).
\begin{enumerate}
	\item Bestimme Basis $B$ von $V$.
	\item Löse das LGS $Bx = v$.
	\item $[v]_B = x$.
\end{enumerate}
\subsection{Abbildungsmatrix (Darstellungsmatrix)}
Lineare Abbildung $f:\mathbb{R}^n \rightarrow \mathbb{R}^m$ \\
Abbildungsmatrix spaltenweise:
$[f] = \begin{pmatrix}f(e_1) & \dots & f(e_n)
\end{pmatrix} $ \\ \\
\textbf{Allgemein} $f:V\rightarrow W$ mit $V, W$ Vektorräume \\
$B=(b_1,\dots,b_n)$ ist eine Basis von $V$, $\exists B^{-1}$. \\
$[f]_B := B^{-1}[f]B$.\\ \\
Gesucht: $x \in\mathbb{R}^n$ mit $f(x) = b$ und $b\in\mathbb{R}^n$.\\
- Löse das LGS $[f]_Bx = B^{-1}b$.\\
Folgende Aussagen sind äquivalent:\\
a) $[f]_B = \begin{pmatrix}
\lambda_1 & & & \\
& \lambda_2 & & \\
& & \ddots & \\
& & & \lambda_n \\ 
\end{pmatrix}$.\\
b) $f(b_1) = \lambda_1b_1, ..., f(b_n) = \lambda_nb_n$.\\ \\

$C=(c_1,\dots,c_n)$ ist eine Basis von $V$. \\
$\Rightarrow [f]_B^C = \begin{pmatrix}
\vert & \vert &  & \vert \\
f(b_1)_C & f(b_2)_C & \cdots & f(b_n)_C\\
\vert & \vert &  & \vert
\end{pmatrix}$ \\
ist die Darstellungsmatrix von $f$ bzgl. $B$ und $C$. \\
''In der j-ten Spalte der Abbildungsmatrix stehen die Koordinaten des Bildes $f(b_j)$ bzgl. der Basis $C=(c_1,\dots,c_m)$'' \\ \\
\textbf{Eigenschaften von $f$ mit Hilfe von $[f]$}
\begin{itemize}\itemsep0pt
\item $\Kern(f) = \{ x\in\mathbb{R}^n : f(x) = 0 \}$
\item $\im(f) = \col([f])$
\item $f$ injektiv, wenn $\Kern([f])=\{0\}$
\item $f$ surjektiv, wenn $\im([f])=\mathbb{R}^m$
\item $f$ bijektiv, wenn $[f]$ invertierbar
\item $f$ ist bijektiv $\Leftrightarrow f$ ist injektiv $\Leftrightarrow f$ ist surjektiv
\end{itemize}

\subsection{Transformationsmatrix}
Transformationsmatrix der Koordinaten von $B$ zu $C$: $_CT_B$\\
\textbf{Regeln und Berechnung:}
\begin{itemize}\itemsep2pt
	\item $_{\mathbb{I}_n}T_B = T_B$: Vektoren der Basis $B$
	\item $_CT_B = {_CT} \cdot T_B$
	\item $(_CT_B)^{-1} = {_BT_C}$
	\item $[v]_C = {_CT_B}\cdot[v]_B$
	\item $C=B\cdot {_CT_B}$
\end{itemize}
\begin{minipage}{\columnwidth}
\section{Diagonalisierung (Eigenwerte und Eigenvektoren)}
Gegeben: Quadratische Matrix $A \in \mathbb{R}^{n\times n}$.\\
Gilt $Av = \lambda v$ mit $v\ne 0$, so nennt man
\begin{itemize}\itemsep0pt
\item $v\in V$ einen \textbf{Eigenvektor} von $A$ zum \textbf{Eigenwert} $\lambda \in \mathbb{R}$ und 
\item $\lambda \in \mathbb{R}$ einen \textbf{Eigenwert} von A zum \textbf{Eigenvektor} $v\in V$
\end{itemize}
Ist $\lambda$ ein Eigenwert von A, so nennt man den Untervektorraum
\begin{itemize}\itemsep0pt
\item $\text{Eig}_A(\lambda) = \{v \in \mathbb{R}^n | Av = \lambda v\}$ den Eigenraum von A zum Eigenwert $\lambda$ und
\item $\dim({\text{Eig}_A(\lambda))}$ die geometrische Vielfachheit des Eigenwerts $\lambda$
\item $\text{geo}(\lambda) = \dim(\text{Eig}_A(\lambda))$ \\
\end{itemize}
\end{minipage}
\begin{minipage}{\columnwidth}
\textbf{Diagonalisieren von Matrizen}\\
$A$ ist diag.bar falls eine invertierbar Matrix $B$ existiert, sodass
\begin{equation*}
D=B^{-1}AB \ \Leftrightarrow \ A=BDB^{-1}
\end{equation*}
und $D$ eine Diagonalmatrix ist.
\begin{itemize}\itemsep0pt
\item Eine Matrix ist genau dann diagonalisierbar wenn $\text{alg}(\lambda)=\text{geo}(\lambda)$ für jeden Eigenwert $\lambda$ von $A$ gilt.
\item Jede Matrix $A \in \mathbb{R}^{n\times n}$ mit $n$ verschiedenen Eigenwerten ist diagonalisierbar.
\item Eine symmetrische Matrix hat nur reelle Eigenwerte und ist diagonalisierbar.
\item Die Determinante einer Matrix ist gleich dem Produkt der Eigenwerte: $\det(A)=\lambda_1\dots\lambda_n$
\end{itemize}
\end{minipage}
\subsection{Rezept: Diagonalisieren}
Gegeben: $A\in \mathbb{R}^{n\times n}$
\begin{enumerate}\itemsep0pt
\item Bestimme das charakteristische Polynom von $A$
\begin{equation*}
p_A(\lambda)=\det(A-\lambda \mathbb{I}_n)
\end{equation*}
\item Charakteristische Polynom $p_A$ in Linearfaktoren zerlegen.
\begin{equation*}
p_A(\lambda)=(\lambda_1-\lambda)^{\nu_1}\dots(\lambda_r-\lambda)^{\nu_r}
\end{equation*}
Es gilt $\nu_1 + \dots + \nu_r=n$ \\
$\lambda_1,\dots, \lambda_r$ sind die Eigenwerte mit algebraischer Vielfachheit $\text{alg}(\lambda_i)=\nu_i$\\
Ist $p_A$ nicht vollständig in Linearfaktoren zerlegbar $\Rightarrow$ A nicht diagonalisierbar!
\item Bestimme zu jeden Eigenwert $\lambda_i$ den Eigenraum $V_i$
\begin{equation*}
V_i=\Kern(A-\lambda_i\mathbb{I}_n)=\text{span}(B_i)
\end{equation*}
Die Vektoren der Basis $B_i$ sind die Eigenvektoren von $\lambda_i$.\\ \\
\textbf{Einfacher:} Der Eigenvektor $v_i$ ist Lösung des homogenen LGS
\begin{equation*}
(A-\lambda_i \mathbb{I}_n)v_i=0
\end{equation*}
$\dim(V_i)=\text{geo}(\lambda_i)$  geometr. Vielfachheit des Eigenwerts $\lambda_i$. \\
Gilt geo$(\lambda_i)\ne\text{alg}(\lambda_i)$ für ein $i$, ist A nicht diagonalisierbar!
\item $B=(v_1 \ \dots v_n)$ setzt sich aus den Eigenvektoren zusammen. \\
$D=\diag(\lambda_1,\dots,\lambda_n)$ ist die Diagonalmatrix der Eigenwerte.
\begin{equation*}
D=B^{-1}AB \ \Leftrightarrow \ A=BDB^{-1}
\end{equation*}
\end{enumerate}
\subsection{Orthogonale Diagonalisierbarkeit}
Eine symmetrische Matrix $A \in \mathbb{R}^{n \times n}$ ist orthogonal diagonalisierbar.
\begin{equation*}
D=Q^{T}AQ \ \Leftrightarrow \ A=QDQ^{T}
\end{equation*}
\textbf{Vorgehen}
\begin{itemize}
\item Eigenwerte und Eigenvektoren bestimmten
\item Eigenvektoren zum selben Eigenwert orthonormalisieren (Gram-Schmidt)
\item Eigenvektoren zu unterschiedlichen Eigenwerten sind bereits orthogonal (ggf. noch normieren)
\item $Q=(b_1 \ \dots b_n)$ setzt sich aus den orthonormalen EV zusammen. \\
$D=\diag(\lambda_1,\dots,\lambda_n)$ ist die Diagonalmatrix der Eigenwerte.
\end{itemize}

\section{QR-Zerlegung}
$A = QR$, wobei $Q$ orthonormale Spalten hat und $R$ oben dreieckig ist.\\
\textbf{Vorgehen}
\begin{itemize}\itemsep0pt
 \item $Q$ berechnen durch Gram-Schmidt mit den Spalten von $A$, \textbf{beginnend bei der ersten}
 \item Die Koeffizienten von $R$ ergeben sich aus den Gram-Schmidt Gleichungen wie folgt: $r_{i,i}=||c_i||^2$ und $r_{i,j}=\frac{\langle v_j,c_i \rangle}{r_{i,i}}$
 \item Alternativ gilt: $R = Q^TA$
\end{itemize}
\section{Kleinstes-Quadrate-Problem}
Für $Ax = b$ lautet die
\textbf{Normalengleichung}
$A^TAx^* = A^Tb$\\
\\
\textbf{Lösen} durch Gauß oder Umstellen: $x^* = (A^TA)^{-1}A^Tb$\\
Für  $A=QR$ lautet die Lösung $x^* = R^{-1}Q^Tb$\\
$\Rightarrow$ optimale Lösung mit minimalem quadratischen Fehler (gibt es immer).
\begin{minipage}{\columnwidth}
\subsection{Rezept: Quadratische Funktion finden}
Gegeben: mehrere Punkte ($x_i, y_i$) im Koordinatensystem\\
Gesucht: Koeffizienten $\mathtt{a}, \mathtt{b}, \mathtt{c}$ für $f(x)=\mathtt{a}\cdot x^2+\mathtt{b}\cdot x+\mathtt{c}\cdot 1=y$\\
\textbf{Vorgehen}
\begin{enumerate}
\item $b = (y_1, y_2, \dots)^\top$ \qquad $A = \begin{pmatrix}
x_1^2 & x_1 & 1 \\[3pt]
x_2^2 & x_2 & 1 \\[3pt]
\vdots & \vdots & \vdots
\end{pmatrix}$ \\
\textit{Falls Ausgleichsgerade gesucht: erste Spalte von $A$ streichen}
\item Lösungsvektor $x^*$ mithilfe von $A$ und $b$ berechnen
\item $x^*=(\mathtt{a}, \mathtt{b}, \mathtt{c})^\top \rightarrow \ f(x)=\mathtt{a}x^2+\mathtt{b}x+\mathtt{c}$
\end{enumerate}
\end{minipage}
%\textbf{Lineare Funktion $(bx+c)$:} erste Spalte von $A$ streichen. $x^*=(\mathtt{b}, \mathtt{c})^\top$
\section{Singulärwertzerlegung}
Bei der Singulärwertzerlegung wird eine beliebige Matrix $A\in \mathbb{R}^{m\times n}$ als Produkt dreier Matrizen $V$, $\Sigma$ und $W$ geschrieben
\begin{equation*}
A=V\Sigma W^\top
\end{equation*}
mit $V\in \mathbb{R}^{m\times m}$, $\Sigma\in \mathbb{R}^{m\times n}$ und $W\in \mathbb{R}^{n\times n}$.\\
$V$ und $W$ sind orthogonal, $\Sigma$ ist eine Diagonalmatrix.
\subsection{Rezept: Singulärwertzerlegung}
Gegeben: $A\in \mathbb{R}^{m\times n}$
\begin{enumerate}\itemsep0pt
\item Bestimme alle Eigenwerte $\lambda_j$ und Eigenvektoren $w_j$ der Matrix $A^\top A\in \mathbb{R}^{n\times n}$ und ordne sie \\ $\lambda_1\ge\lambda_2\ge \dots \ge \lambda_r>\lambda_{r+1}=\dots=\lambda_n=0$ mit $r\le n$
\item Bestimme eine ONB des $\mathbb{R}^n$ aus den Eigenvektoren $w_j$ und erhalte $W=\begin{pmatrix} 
w_1 &\dots & w_n
\end{pmatrix} \in \mathbb{R}^{n\times n}$
\item Die Singulärwerte sind $\sigma_j=\sqrt{\lambda_j}$ \qquad $j=1,\dots,\min\{m,n\}$
\begin{equation*}
\Sigma=\begin{pmatrix}
\sigma_1 & & & 0 & \dots & 0\\
 & \ddots & &  \vdots &  &  \vdots\\
& & \sigma_m & 0 & \dots & 0\\
\end{pmatrix} \in \mathbb{R}^{m\times n}
\qquad m<n
\end{equation*}
\begin{equation*}
\Sigma=\begin{pmatrix}
\sigma_1 & & \\
 & \ddots & \\
& & \sigma_n \\
0 & \dots & 0\\
\vdots &  &  \vdots\\
0 & \dots & 0\\
\end{pmatrix} \in \mathbb{R}^{m\times n}
\qquad m>n
\end{equation*}
\item Bestimme $v_1,\dots,v_r$ aus $v_i=\frac{1}{\sigma_j}Aw_j$ für alle $j=1,\dots,r$ (alle $\sigma_j\ne 0$)
\item Falls $r<m$ ergänze $v_1,\dots,v_r$ zu einer ONB, bzw. zu $V=\begin{pmatrix}
v_1 & \dots & v_m
\end{pmatrix}$
orthogonal.
\item $A=V\Sigma W^\top$
\end{enumerate}
\begin{minipage}{\columnwidth}
\subsection{Kompression}
\begin{itemize}\itemsep0pt
\item Rang k Matrix $A_{(k)}=V\Sigma_{(k)}W^\top$. Die Matrix $\Sigma_{(k)}$ enthält nur die Singulärwerte $\sigma_1, ...,\sigma_k$.\qquad ($\sigma_{k+1}, ...,\sigma_{n/m}$ mit 0 ersetzen).\\
\item Frobenius-Norm $|| A ||_F = \sqrt[]{\sum_{i=1}^{m} \sum_{j=1}^{n} {a^{2}_{i,j}}}$\\
\item Es gilt: $|| A-A_{(k)} ||_F \leq || A-B ||_F$ mit $B$ als eine beliebige Matrix des $\mathbb{R}^{m\times n}$ mit $rang(B) \leq rang(A_{(k)}) = k$
\item Speicheraufwand einer Matrix $A\in \mathbb{R}^{m\times n}$: $rang(A) \cdot (m+n)$
\end{itemize}
\end{minipage}
\subsection{Exkurs: Hauptkomponentenanalyse (PCA)}
Gegeben: Matrix $W$ der Eigenvektoren von $\mathbf{A\cdot A^\top}$\\
Gesucht: Projektionen der Spalten von $A$ auf den Unterraum $U_{(k)}$ als Koordinaten bzgl. der ONB der Eigenvektoren von $AA^\top$\\
Vorgehen: 
\begin{enumerate}
	\item Normiere $w^1, \dots , w^k$
	\item $\overline{w}=(w^1 \dots w^k)\in \mathbb{R}^{m\times k}$
	\item Die Koordinatenvektoren $y^1, \dots , y^k \in \mathbb{R}^{k}$ die Vektoren $x^1,\dots , x^n$ bzgl. der Basis $\{w^1, \dots , w^k\}$ von $U_{(k)}$ ausdrücken, sind die Spalten von\\
	$\overline{w}^\top A=\begin{pmatrix}
	(w^{1})^{\top} \\
	\vdots \\
	(w^{k})^{\top}
	\end{pmatrix}
	\cdot A$
\end{enumerate}
\section{Lineare Differentialgleichungen \& Rekursive Folgen}
\subsection{Lösen einer linearen Differentialgleichung}
Gegeben: $y'(t) = \lambda y(t)$, mit $y(0) = c$\\
Lösung: $y(t) = ce^{\lambda t}$ mit $c \in \mathbb R$\\
\subsection{Lösen eines Systems linearer Differentialgleichungen}
Gegeben: $y'(t) = Ay$, mit $y_1(0) , ..., y_n(0)$\\ 
Wobei $A \in \mathbb R^{nxn}, y_1(0) , ..., y_n(0) \in \mathbb R, y,y'\in\mathbb R^n$.\\
Lösung:\\
\begin{enumerate}
	\item Eigenwerte und Eigenvektoren von $A$ bestimmen.
	\item $\begin{pmatrix}
	y_1(t)\\
	\vdots\\
	y_n(t) \\
	\end{pmatrix}
	= c_1e^{\lambda_1t}v_1+...+c_ne^{\lambda_nt}v_n$.
	\item Anfangswerte einsetzen und Werte für $c_1$, bis $c_n$ bestimmen.
\end{enumerate}
\ \\
\subsection{Rekursive Folgen}
Gegeben: $x_{n+1} = {\alpha} x_n + {\beta} x_{n-1}$, Anfangswerte: $x_0$, $x_1$\\
Lösung:\\
\begin{enumerate}
	\item Matrix A bestimmen. 
	\begin{equation*}
	A=\begin{pmatrix}
	{\alpha}&{\beta}\\
	1&0
	\end{pmatrix}
	\end{equation*}
	\item LGS aufstellen.
	\begin{equation*}
	\begin{pmatrix}
	x_{n+1}\\
	x_{n}\end{pmatrix}=\begin{pmatrix}
	{\alpha}&{\beta}\\
	1&0
	\end{pmatrix}
	\begin{pmatrix}
	x_{n}\\
	x_{n-1}\end{pmatrix}
	\end{equation*}
	\item Diagonalisieren: $A=SDS^{-1}$
	\item Anfangswerte $x_0, x_1$ einsetzen und $x_n$ berechnen
	\begin{equation*}
	\begin{pmatrix}
	x_{n+1}\\
	x_{n}\end{pmatrix}=SD^{n}S^{-1}
	\begin{pmatrix}
	x_{1}\\
	x_{0}\end{pmatrix}
	\end{equation*}
\end{enumerate}
\section{Definitheit und Quadratische Funktionen}
\subsection{Definitheit}
Eine symmetrische Matrix $A \in\mathbb R^{n\times n}$ mit den Eigenwerten (EW) $\lambda_1, ..., \lambda_n$ heißt:
\begin{itemize}\itemsep0pt
\item positiv definit: EW positiv $<=> v^{\top}Av> 0,\quad\forall v \in\mathbb R^n\backslash \{0\}$
\item negativ definit: EW negativ $<=> v^{\top}Av< 0,\quad\forall v \in\mathbb R^n\backslash \{0\}$
\item positiv semidefinit: EW $\geq 0 <=> v^{\top}Av\geq 0,\quad\forall v \in\mathbb R^n$
\item negativ semidefinit: EW $\leq 0 <=> v^{\top}Av\leq 0,\quad\forall v \in\mathbb R^n$
\end{itemize}
\subsection{Quadratische Funktionen}
\textbf{Form:} $f(x)=x^{\top}Ax+b^{\top}x+c = \langle x,Ax \rangle + \langle b,x \rangle + c$\\
Berechnen von Extrempunkten:
\begin{itemize}\itemsep0pt
\item positiv definit: Minimum bei $x^{*}=-\frac{1}{2}A^{-1}b$
\item negativ definit: Maximum bei $x^{*}=-\frac{1}{2}A^{-1}b$
\item positiv/negativ semidefinit: Existenz von Extremum hängt von Lösbarkeit des LGS $2Ax=-b$ ab. (nicht eindeutig!)
\end{itemize}

% Ende der Spalten


% Dokumentende
% ======================================================================
\end{document}
